\documentclass[12pt,twoside,a5paper]{amsbook}

\usepackage[usenames,dvipsnames]{color}
\usepackage{epsfig,exscale}
\usepackage[latin1]{inputenc}
\usepackage{hyperref}
 
\usepackage{setspace}
\onehalfspacing

\usepackage[includeheadfoot,
top=20mm,
left=30mm=,
right=20mm,
bottom=15mm
]{geometry}

\usepackage{titlesec}

\renewcommand{\chaptername}{}
\renewcommand{\thechapter}{\Roman{chapter}}
\renewcommand{\contentsname}{�ndice}

\titleformat{\chapter}[frame]{\normalfont\LARGE\bfseries}
{\ \thechapter\ }{2.0ex}{\LARGE\sc}

\newcommand{\dlg}{\noindent $\bullet$ }

\usepackage{contour}
\contourlength{0pt} %how thick each copy is
\contournumber{1}  %number of copies

\newcommand{\fL}[1]{
\noindent
\contour{black}{\bfseries \Huge \textcolor{white}{#1}}
\hspace{-1.5ex}
}

\hypersetup{
pdftitle={Cuentos para ir a dormir},
pdfsubject={cuentos},
pdfauthor={Ariel Narv�ez},
pdfkeywords={cuentos, Emma, �risz}
}

\title[Cuentos]{CUENTOS PARA IR A DORMIR} 
\author[Ariel Narv�ez]{Ariel Narv�ez}

\begin{document}
\maketitle

\hfill ...a Emma e �risz

\tableofcontents

\chapter[El Cocodrilo y las Hormigas]{El Cocodrilo y las \\ Hormigas}
\fgr{0.75}{cocodrilo_1.png}

Un d�a de verano el Cocodrilo se levant� muy temprano y tras un muy
buen desayuno se dispuso a pintar su casa. 
%
�sta no era muy grande pero ten�a un gran jard�n con muchas
plantas. 
%
El Cocodrilo ten�a preparadas las brochas y todos los colores
para las distintas partes de la casa: El techo lo iba a pintar de
color rojo; las ventanas, azules; la puerta, lila; las paredes, verdes
y la cerca que rodea todo el jard�n, blanca.

El Cocodrilo estuvo pintando sin descanso durante toda la ma�ana y
finalmente poco antes del mediod�a termin� toda la casa. 
%
Lav� primero todas las brochas y luego las guard� ordenadas junto con
las pinturas. 
%
Se lav� muy bien las manos y fue a cocinar su almuerzo.

El almuerzo consisti� en unos ricos tallarines con salsa y para el
postre una manzana. 
%
El Cocodrilo decidi� ir a descansar a la orilla del r�o y comerse el
postre all� mejor. 
%
\fgr{0.8}{cocodrilo_2.png}
%
All� �l ten�a una mesa y una silla en la que normalmente disfrutaba de
tomar sol en el verano. 
%
El Cocodrilo dej� la manzana sobre la mesa, se recost� y disfrut� del
sol.
%
Luego de una reponedora siesta, el Cocodrilo se meti� al r�o a nadar.
%
Aunque el r�o ten�a un fuerte caudal, �l no ten�a problemas ya que con
la ayuda de su fuerte cola con facilidad se sobrepon�a a la corriente
y pod�a cruzar el r�o de un lado a otro.
%
\fgr{0.85}{cocodrilo_3.png}

Al terminar de nadar y disponerse a comer la manzana, el Cocodrilo
se encontr� con una sorpresa: la manzana ya no se encontraba sobre la
mesa. 
%
El Cocodrilo empez� a buscarla y r�pidamente la encontr� bajo �sta,
pero algo raro ocurr�a, la manzana se mov�a lentamente. 
%
Al acercarse y mirar detenidamente, el Cocodrilo se dio cuenta que
unas Hormigas se la llevaban.
%
El Cocodrilo les dijo que se la devolvieran a lo que las Hormigas le
res\-pon\-die\-ron:
%
\fgr{0.75}{cocodrilo_4.png}
%
\dlg \xclm{Nosotras somos muchas y tenemos hambre} \\
El Cocodrilo muy enojado les respondi�: \\
\dlg Si tienen tanta hambre vayan ustedes mismas a buscar
manzanas al �rbol y no me quiten la m�a. \\
Las Hormigas respondieron muy sorprendidas:\\
\dlg \prgnt{Cu�l �rbol} \\
Las Hormigas no conoc�an el �rbol as� que muy curiosas le
preguntaron al Cocodrilo: \\
\dlg \prgnt{D�nde est� el manzano} \\
El Cocodrilo aunque estaba a�n muy enojado les respondi� amablemente y
les dijo que simplemente ellas ten�an que seguir el r�o y caminar unos
diez minutos y se encontrar�an con el manzano.

Las Hormigas le devolvieron la manzana al Cocodrilo y partieron
r�pidamente a buscar manzanas. 
%
Cuando llegaron al �rbol todas las Hormigas se subieron y sacaron una
manzana para cada una.
%
Adem�s buscaron la manzana m�s roja y grande del �rbol y se la
llevaron de regalo al Cocodrilo para agradecerle por ense�arles la
ubicaci�n del manzano y para pedirle perd�n por tratar de llevarse su
manzana.
%
\fgr{0.75}{cocodrilo_5.png}


\chapter{El Libro Benjamin}
Era un d�a muy fr�o de invierno, el Libro Benjam�n caminaba por la
ciudad ya en oscuridad en busca de su familia. �l no conoc�a la ciudad
pero sab�a que su familia se encontraba ac�. El Libro Benjam�n se
decidi� a preguntar por ayuda. Debido al fr�o no hab�a mucha gente en
las calles.

El Libro Benjam�n decidi� entrar a un edificio que ten�a dos grandes
camiones rojos estacionados en su interior. Muy amablemente se
present� y pregunt� si posiblemente estaba su familia en el lugar. En
el edificio le respondieron que eso era un Cualtel de Bomberos y que
su familia no estaba all�. El Libro Benjam�n muy curioso pregunt�: \\
\dlg Qu� hacen los Bomberos? \\ Los Bomberos amablemente le
respondieron, que ellos estaban siempre preparados para ir r�pidamente
a apagar cualquier incendi� que pusiera en peligro la casa de alguna
familia de la ciudad. 

El Libro Benjam�n se despidi� de los bomberos y sigui� su camino,
entro en la siguiente puerta abierta que encontr�. Pregunt� si ah� se
encontraba su familia a lo que le respondieron que no, que eso era un
Carnicer�a. El Libro Benjam�n pregunt�: \\ 
\dlg Qu� era eso? \\ 
El se�or Carnicero le respondi� que era un negocio donde se vend�a
carne de Pollo, Cerco y Vacuno. Adem�s vendemos otros productos como
Longaniza y Salchich�n. 

El Libro Benjam�n sigui� su camino y luego de caminar algunas cuadras
entr� nuevamente a otra lugar para preguntar por su Familia. Ah� le
respondieron que eso era una Panader�a y su Familia no estaba ah�. El
Libro Benjam�n muy curioso pregunt�: \\
\dlg Qu� es una Panader�a? \\
Le respondieron: \\
\dlg Ac� nos levantamos todos los d�as muy temprano a preparar pan,
para que las familias de toda la ciudad vengan a comprarlo. Adem�s
hacemos K�chen y Tortas para los cumplea�os.

Era ya muy tarde, la ciudad estaba fr�a y oscura y el Libro Benjam�n
aun no encontraba a su Familia. Ya sin muchas esperanzas entr� al
siguiente lugar abierto y alumbrado a preguntar por su Familia. Le
respondieron que ah� no estaban y que eso era un Bazar, donde se
vend�an L�pices, Cuadernos, Tarjetas, Sobres, Pegamento, Tijeras,
Carpetas y Corcheteras; entre muchas otros productos. El Libro
Benjam�n les pregunt� si le pod�an dar un consejo para encontrar a su
Familia. La Se�ora del Bazar le dijo que posiblemente en la Biblioteca
de la Ciudad que estaba muy cerca.

El Libro Benjam�n se despidi� r�pidamente y se fue corriendo a la
Biblioteca. Al encontrarla entr� r�pidamente pero al notar el gran
silencio que ah� reinaba, freno su paso. Dentro de la Bibliteca hab�a
mucha gente muy concentrada leyendo. Pronto identific� a su Mam�, {\sl
  La Se�ora Enciclopedia} y luego vi� a su Pap�, {El Se�or
  Diccionario}. Se acerc� a ambos y les dio un fuerte abrazo, tambi�n
a su hermana la Novela y a su primo el Atlas.

Luego una busqueda tan larga, finalmente el Libro Benjam�n encontr� a
su Familia y no se volvi� a separar nunca m�s de ellos.


\chapter{Las Hormigas y el Topo}
Durante todo el verano las Hormigas se fueron a la playa. Algunas
jugaban volleyball, otras f�ltbol, otras paletas, otras nadaban, otras
contru�an castillos de arena y las restantes simplemente disfrutaban
del sol.

Cuando y el verano se acababa, la jefa de las Hormigas llam� a una
reuni�n a la cual asistieron todas las Hormigas. La jefa con vez de
mando dijo: \\
\dlg  Hermanas, nuestras vacaciones en la playa se terminan
hoy. A partir de ma�ana comenzamos la recolecci�n de alimentos para el
duro invierno que viene. Tenemos dos meses para la recolecci�n y
despu�s debemos construir un hogar para nosotros y nuestra comida.

Todas las Hormigas escuchaban muy atentamente. La Hormiga secretaria
tomaba nota de todo lo que dec�a la jefa. La jefa de la Hormigas
segu�a con su plan: \\ 
\dlg  Los alimentos necesarios son: trigo, maiz, arroz, manzana,
peras, papas, leche, miel, porotos, lentejas, y arvejas. Para eso nos
dividiremos en escuadras que se dedicaran a recolectar cada uno de
estos alimentos. Mi secretaria ac� tiene las listas con las escuadras. 

Para terminar la jefa le dijo al resto de las Hormigas: \\
\dlg  Ma�ana comenzamos my temprano as� que hoy disfruten lo m�s
posible de la playa por que ser� el �ltimo y luego se nos viene un
duro trabajo.

Al d�a siguiente y durante los siguientes las Hormigas trabajaron
duro. Luego de los dos meses de recolecci�n, las Hormigas ya ten�an
suficiente alimentos para el invierno y deb�an ahora iniciar la
construcci�n de su hogar. Por alguna extra�a raz�n, ese a�o la nieve
se adelant� y las Hormigas a�n no ten�an su hogar. Esto era un
problema muy grave. La jefa de las Hormigas les dijo a sus
compa�eras: \\
\dlg Debemos de encontrar pronto un hogar para nosotras y nuestra comida...

En ese momento el papagayo que miraba desde arriba interrumpi� a la
jefa y le dijo: \\

\dlg Seg�n lo que yo s�, el Topo agrand� su casa durante el verano y tiene,
por tal, unas piezas libres que les podr�an ser �tiles. Yo ir�
r�pidamente a preguntarle. \\ 
El papagayo sali� volando r�pidamente y
se encontr� con el Topo explic�ndole el problema de las Hormigas. El
Topo respondi� r�pida y amablemente: \\
\dlg Ning�n problema, que vengan a vivir conmigo durante el invierno.

El Papagayo vol� de vuelta y le di� la buena noticia a la Hormigas. La
jefa de la Hormigas, r�pidamene organiz� el viaje hasta la casa del
Topo e iniciaron el viaje. Al llegar a la casa del Topo las Hormigas
saludaron y agradecieron la amabilidad del Topo. El Topo les
respondi�: \\
\dlg Gracias a ustedes que me van a acompa�ar durante el invierno. Yo
vivo solo, trabajo durante todo el d�a afuera y llego muy cansado en
la tarde. Ahora estaran ustedes esper�ndome, podemos comer juntos y
despu�s tambi�n jugar a las cartas.

Ese invierno fue muy entretenido para las Hormigas y el Topo. Las
Hormigas usando toda la comida que hab�an recolectado le cocinaban al
Topo una rica cena para cuando �ste llegaba cansado de su trabajo.






\chapter{El Mensaje}
Muy temprano en la ma�ana, antes que el Gallo cantara, al Dinosaurio,
al Elefante y a la Girafa los despertaron para que
partieran r�pidamente con la misi�n de entregar un mensaje al Le�n.
%
�ste hab�a salido de viaje hace ya casi un mes y por distintos
problemas no hab�a podido regresar aun.

El Dinosaurio, el Elefante y la Girafa no tuvieron tiempo para
preparar nada y salieron r�pidamente con una mochila cargada con un
par de manzanas, unos panes y una botella con agua.
%
Debido a la distancia que se encontraba el le�n, el Dinosaurio, el
Elefante y la Girafa estimaron que les tomar�a todo el d�a caminado
hasta llegar donde el le�n.
%
En el trayecto deb�an cruzar la selva, la sabana y un r�o.
%
Aunque ninguno de los tres sab�a nadar no ten�an miedo del r�o ya que
los tres eran animales muy grandes, por tal, no ser�a necesario nadar
para cruzarlo.

Al poco andar por la selva se encontraron con los chimpanc�s, los cuales
muy ruidosos saludaron.
%
Luego se encontraron con el gorila, que adem�s de saludar les pregunt�: \\
\dlg  \prgnt{Ad�nde van con tanta prisa} \\
El Dinosaurio respondi�:
\dlg Llevamos un mensaje al Le�n, as� que tenemos que caminar 
muy r�pido para llegar antes que oscurezca. \\
\dlg Mucha suerte en su misi�n les dese� entonces. \\
Les dijo el gorila.

Luego de un par de horas caminando, el Dinosaurio, el Elefante y la
Girafa decidieron hacer la primera pausa para comerse las mazanas y
tomar un poco de agua.
%
Retomaron el rumbo y pronto cruzaron la selva y se adentraron en la
sabana.
%
Ac� se encontraron con los ant�lopes y las zebras que muy
cordialmente saludaron.
%
Ya, a la distancia pod�an divisar el r�o y luego de un par de minutos
caminando llegaron a su orilla.
%
El r�o no se ve�a muy profundo pero si con un caudal muy fuerte.
%
El Dinosaurio, el Elefante y la Girafa no ten�an miedo del �ste y se
adentraron en el agua.
%
Al cabo de unos metros sintieron la fuerza del caudal y luego de unos
segundos, cayeron al agua y se fueron por el r�o arrastrados por el
agua.
%
Ninguno de ellos se pod�a poner en pi� nuevamente y rodaban r�o abajo.
%
Por suerte el Cocodrilo que viv�a en el r�o observ� todo lo acontecido
y r�pidamente se met�o al agua y con su fuerte cola avanaz�
r�pidamente, tom� a la Girafa y la cruz� hasta la otra orilla del r�o.
%
Se meti� nuevamente y ahora sac� al Elefante y por �ltimo y sac� al
Dinosaurio.
%
�ste era tan grande que le cost� un esfuerzo enorme arrastrarlo hasta
la otra orilla.

Una vez en la otra orilla, secos, el Cocodrilo les dijo que aunque
ellos eran muy grandes, ten�an que tener cuidado con lo r�os.
%
Muchos no son profundos pero el caudal puede hacer que uno pierda el
equilibrio y se caiga, como les acababa de ocurrir a ellos.
%
Por otra parte, el Dinosaurio, el Elefante y la Girafa se dieron
cuenta que producto de su ca�da al agua, se les hab�a perdido el pan
que tra�an.
%
Ahora ya no ten�an nada m�s para comer y todav�a les quedaba mucho por
caminar.

Caminaron ahora sin descanzo por la sabana, ya se hac�a de noche y
todos los animales se iban a dormir. 
%
El atardecer en la sabana era muy bonito con un sol grande y rojo que
se escond�a en el horizonte acompa�ado de un cielo naranjo.
%
Siguieron su camino y ya hab�an caminado muchas horas en la oscuridad
cuando llegaron donde el Le�n y finalmente pudieron entregarle el
mensaje.
%
El Le�n abri� la carta y pudo leer:
%
\begin{center}
\begin{tabular}{|r|}
\hline
{\sc Tu hijo ha nacido hoy sano y fuerte.} \\
{\sc La Leona.}\\
\hline
\end{tabular}
\end{center}

El Le�n salt� de felicidad y mand� a preparar una gran comida para sus
amigos mensajeros.
%
El Dinosaurio, el Elefante y la Girafa despu�s de un largo d�a
caminando, finalmente pudieron descanzar con el grato sabor de haber
cumplido la misi�n.


\chapter{El Topo y el Mono}
Luego de un largo invierno, en el cual el Topo se refugi� en su cueva
para as� mantenerse calentito, llegaron los primeros d�as c�lidos de
la primavera y, por tal, tiempo de salir de la madriguera.

El Topo, lo primero que hizo fue abrir la entrada a su
madriguera para que entrara luz. 
%
Luego, inspirado por el fresco aire que pudo respirar decidi� limpiar
bien su madriguera ya que durante todo el invierno estuvo durmiendo y
mucho polvo se acumul� en todas partes.

Durante toda la ma�ana el Topo estuvo limpiando su casa, limpi� las
ventanas, trape� el piso de la cocina y el ba�o, sacudi� los cojines
del sill�n, barri� el piso y luego lo encer�, aspir� la alfombra y
finalmente con el plumero sac� el polvo de los muebles. 
%
Con la limpieza a fondo llen� varias bolsas de basura que las dej�
afuera de su casa para ir m�s tarde a botarla al contenedor.

Luego del arduo trabajo, el Topo estaba listo para descansar. Sali� de
su madriguera y se sent� a la sombra de un �rbol. 
%
Llev� junto con �l una rica sand�a para comer mientras descansaba.

Por otra parte, en la cima de ese mismo �rbol unos minutos antes hab�a
subido el Mono a leer y para comer hab�a llevado un pl�tano consigo.
%
Cuando el Mono se decidi� a comer el pl�tano, lo pel� por completo y
dej� la c�scara colgada de una rama para luego cuando bajara ir a
botarla a la basura.
%
Con una peque�a brisa las ramas del �rbol se movieron y la c�scara
cay�.
%
El Mono no le di� mayor importancia y se dijo a si mismo que cuando
bajara recoger�a la c�scara y la ir�a a botar a la basura.

La c�scara fue cayendo chocando de rama en rama para finalmente caer
exactamente sobre la cabeza del Topo que descansaba junto con su
sand�a. 
%
El Topo muy enfurecido mir� a todos lados buscando desde d�nde lleg�
la c�scara, no encontr� a nadie, luego mir� hacia arriba y pudo ver al
Mono sentado en la cima del �rbol leyendo un libro y comiendo
pl�tano.
%
El Topo se dijo a si mismo: \\
\dlg Mono cochino, botando las c�scaras en cualquier parte y no 
en el basurero. \\
Luego lo llam�: \\
\dlg \xclm{Mono} \xclm{Mono} \xclm{ven, baja del �rbol} \\
Muy �gil, el Mono baj� del �rbol y salud� al Topo: \\ 
\dlg \prgnt{Qu� pasa Topo} \\ 
El Mono todav�a muy enojado le dice: \\ 
\dlg Mira, la c�scara de pl�tano que tu botaste desde arriba me cay� a
m� en la cabeza. \\
\dlg Esa c�scara yo la dej� sobre una rama para botarla m�s tarde a la
basura y el viento la hizo caer. \\
Le respondi� el Mono. Agregando adem�s: \\
\dlg Mil disculpas amigo Topo, nunca fue mi intenci�n tirar la c�scara 
de pl�tano desde all� arriba. \\
\dlg Para que me perdones yo mismo to voy a ayudar a dejar al contenedor 
las bolsas de basura que tienes ah�.

El Mono r�pidamente llev� todas las bolsas de basura la contenedor. Al
volver, el Topo le dio al Mono un pedazo de su sand�a mostrando as�
que el malentendido con la c�scara de pl�tano ya estaba superado y
segu�an siendo amigos.




\chapter{La Manzana}
Un d�a iban caminando juntos por el campo tres animales muy grandes:
un Elefante, una Girafa y un Dinosaurio. Iban de lo mas felices
conversando cuando se dieron cuenta que en frente de ellos
hab�a un manzano con la �ltima de sus manzanas en el tope de su frondosa
copa.

El primero en decir que la manzana le pertenec�a fue el Elefante, que
dijo de inmediato:\\
\dlg  Esa manzana es m�a, la voy a sacar y me la voy a comer. \\
La Girafa y el Dinosaurio al un�sono replicaron la misma respuesta: \\
\dlg  Yo tambi�n la puedo sacar y me la como.

En resumen, los tres querian la manzana y cada uno estaba muy seguro
que debido a su tama�o no ser�a un problema poder sacala aunque la
manzana en realidad se ubicaba a mucha latura. Luego de discutir el
tema llegaron a un acuerdo: Ser�a la Jirafa la primera en intentar
sacar la manzana, luego vendr�a el Elefante y por �ltimo el Dinosaurio
que era el m�s grande de los tres y de seguro sacar�a la manzana.

La Jirafa se acerc� al �rboly con estir� su largo cuello para poder
alcanzar la manzana. Aun con su mejor esfuerzo, tratando incluso de
pararse s�lo en sus patas traseras, la Jirafa no fue capaz de alcanzar
la manzana. Al darse por vencida, le dio el turno al Elefante. �ste se
acerc� se par� en sus patas traseras y estir� su larga y fuerte trompa
pero no alcanzaba la manzana. Incluso con su trompa intent� zamarrear
las ramas para poder as� botar la manzana. Esto tampoco result�.

Finalmente se acerc� al manzano el Dinosaurio muy seguro de sacar la
manzana, estir� su largo cuello y lo dirigi� haci� la manzana pero
para su sorpresa no la pudo alcanzar. Luego de intentarlo varias veces
se convenci� de que le era imposible.

Los tres m�s grandes animales se encontraban completamente
sorprendidos de que ninguno de ellos fue capaz de sacar la manzana. 

En ese momento se acerca al manzano un peque�o mono que empieza a
escalar y saltar de rama en rama subiendo r�pidamente a la copa del
�rbol. Con la mirada at�nita del Elefante, la Jirafa y el Dinosaurio;
el peque�o mono fue capaz de sacar la manzana y muy feliz bajar del
�rbol comi�ndosela. Los saludo afectuosamente y tan repent�namente
como lleg�, se fue. 

El Elefante, la Jirafa y el Dinasaurio al un�sono se rieron de lo que
hab�a pasado y siguieron su camino.


\chapter{El Grillo y el Viol�n}
Desde ya hace muchas generaciones, la familia del Grillo era conocida
por su m�sica. Por tal, al peque�o Grillo su pap� le ense�o a tocar el
viol�n. As�, cuando �l estuviera viejo, el peque�o Grillo pudiera
heredar su trabajo. El trabajo consist�a en tocar el viol�n todas las
noches y as� ayudar a los animales a dormir.

El tiempo pas� muy r�pido y el pap� Grillo se hizo viejo y no ten�a
las fuerzas necesarias para salir a tocar todas las noches el viol�n.
Para que continuara la tarea el Grillo recibi� de regal� el mejor
viol�n. La rutina era siempre la misma, se iniciaba tocando en el
gallinero para hacer dormir a todas las gallinas y sus pollitos, luego
se iba al chiquero donde se hac�a dormir a todos los chanchos y sus
peque�os lechones. Se contiuaba en el establo donde se hac�a dormir a
los caballos y vacas y a sus peque�os potrillos y terneros. Por �ltimo
se hac�a dormir a las ovejas y se iba a tocar al lado de la ventana de
la pieza donde dormian los hijos del granjero para que �stos tambi�n
tuvieran dulces sue�os.

El Grillo estaba muy feliz de realizar esta linda tarea todos los
d�as, todas las semanas preparabas nuevas canciones para que los
animales durmieran muy bien y pudieran as� poder iniciar el nuevo d�a
llenos de energ�a y del mismo modo para que los peque�os pollitos,
lechones, potrillos, terneros y los hijos del Granjero crecieran
fuertes y sanos. El trabajo del Grillo era muy importante en el campo
y todos los animales le ten�an un gran afecto.

Un d�a el Grillo iba a iniciar su trabajo pero no pod�a encontrar el
viol�n, y sin �l no pod�a salir a tocar. Adem�s ese viol�n era muy
importante para �l por que era el favorito de su pap� y �ste se lo
hab�a regalado. Lo busc� por todas partes y no lo pudo
encontrar. Tanto lo busc� que se hizo tarde, decidi� que ese d�a ya no
saldr�a a tocar y en la ma�ana continuar�a la busqueda. Al otro d�a
muy temprano muchos animales se acercaron a preguntar porqu� no hab�a
tocado la noche anterior. Al explicarles el Grillo, los animales se
sumaron a la busqueda del viol�n.

Pasaron cuatro d�as y el viol�n no aparec�a, muchos animales se ve�an
muy cansados, necesitaban de la m�sica del Grillo para poder dormir
bien y as� reponer las energ�as. En la granja no se hablaba de otro
tema, todos los animales estaban muy preocupados y se organizaron para
todos iniciar una busqueda en todas partes hasta encontrar el viol�n,
la cual denominaron: {\sl Operaci�n Rastrillo}.

Iniciaron la busqueda en el gallinero, revizaron todos los rincones,
luego fueron al chiquero, que era el lugar donde viv�an los chanchos y
estaba muy desordenado pero no encontraron el viol�n. Luego al igual
que la rutina del Grillo todas las noches, todos los animales, fueron
al establo y revizaron en todas partes. Para finalizar fueron a la
casa del Grillo y buscaron sin �xito. Para finalizar se reunieron
todos los animales en el centro del patio. En ese momento lleg� el
Granjero con el viol�n en sus manos, �l lo hab�a tomado y llevado a la
ciudad para que lo restauraran con un nuevo barniz adem�s de
colocarles cuerdas nuevas. Todos los animales saltaron de alegr�a al
ver nuevamente el viol�n, especialmente el Grillo debido a que �ste
era un regalo especial de su Pap�.

Ese d�a el Grillo toc� su mejores canciones y todos los animales
durmieron profundamente.





\end{document}

