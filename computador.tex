El Cocodrilo se hab�a integrado desde hace mucho tiempo al mundo de la
computaci�n.
%
Disfrutaba mucho de mantener contacto con sus amigos animales de todo
el mundo, por ejemplo: con el Oso Panda de China, el Canguro de
Australia, el Camello de Persia, el Elefante de India, el Mono Tit� de
Brasil y el C�ndor de Chile.

Un d�a, mientras conversaba con el Oso Panda, �ste le avis� que hab�a
un problema y ya no pod�a ver su imagen.
%
El Cocodrilo se despidi� r�pidamente de su amigo para revisar qu�
problema pod�a tener la c�mara.
%
Lo primero que hizo fue desconectarla y volver a conectarla, pero la
c�mara no funcionaba.
%
Luego apag� y volvi� a encender el computador, pero la c�mara tampoco
volvi� a funcionar.

El Cocodrilo decidi� entonces llamar al servicio t�cnico.
%
Al cabo de unos minutos, son� el timbre.
%
Al abrir la puerta, el Cocodrilo se encontr� con una sorpresa, el
servicio t�cnico eran un grupo de Hormigas.
%
Ellas se presentaron y r�pidamente se dirigieron a revisar la
c�mara.
%
Se pusieron sus peque�os cascos con luz y se metieron adentro, ah�
estuvieron durante unos minutos, durante los cuales el Cocodrilo lo
�nico que escuchaba desde afuera era el murmullo que hac�an las
Hormigas trabajando en el interior de la c�mara.

Al salir, las Hormigas le informaron que el problema de la c�mara era
muy simple, s�lo era necesario cambiar un transistor que se hab�a
quemado.
%
Las Hormigas y el Cocodrilo fueron juntos a comprar uno nuevo a la
tienda. 
%
Al volver con el transistor nuevo, las Hormigas se volvieron a colocar
sus cascos y se metieron nuevamente dentro de la c�mara.
%
Desde fuera el Cocodrilo nuevamente s�lo escuchaba el murmullo.
%
Al rato, las Hormigas salieron con el transistor quemado e ingresaron
nuevamente a la c�mara con el transistor nuevo.
%
Luego de instalarlo las Hormigas se despidieron con la tarea cumplida.

El Cocodrilo volvi� a conectar la c�mara y encendi� su computador.
%
Ahora todo funcionaba perfectamente. 
%
De inmediato llam� nuevamente al Oso Panda para continuar la
conversaci�n pendiente.
